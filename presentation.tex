% !TEX TS-program = pdflatex
% !TEX encoding = UTF-8 Unicode

% This file is a template using the "beamer" package to create slides for a talk or presentation
% - Giving a talk on some subject.
% - The talk is between 15min and 45min long.
% - Style is ornate.

% MODIFIED by Jonathan Kew, 2008-07-06
% The header comments and encoding in this file were modified for inclusion with TeXworks.
% The content is otherwise unchanged from the original distributed with the beamer package.

\documentclass{beamer}


% Copyright 2004 by Till Tantau <tantau@users.sourceforge.net>.
%
% In principle, this file can be redistributed and/or modified under
% the terms of the GNU Public License, version 2.
%
% However, this file is supposed to be a template to be modified
% for your own needs. For this reason, if you use this file as a
% template and not specifically distribute it as part of a another
% package/program, I grant the extra permission to freely copy and
% modify this file as you see fit and even to delete this copyright
% notice. 


\mode<presentation>
{
  \usetheme{Warsaw}
  % or ...

  \setbeamercovered{transparent}
  % or whatever (possibly just delete it)
}


\usepackage[english]{babel}
% or whatever

\usepackage[utf8]{inputenc}
% or whatever

\usepackage{times}
\usepackage[T1]{fontenc}
% Or whatever. Note that the encoding and the font should match. If T1
% does not look nice, try deleting the line with the fontenc.


\title[Short Paper Title] % (optional, use only with long paper titles)
{Reevaluating The Adversary Model for DNS Security}

\subtitle
{Depending on The Kindness of Strangers} % (optional)

\author[Author, Another] % (optional, use only with lots of authors)
{R.~Harrison \and B.~Benshoof}
% - Use the \inst{?} command only if the authors have different
%   affiliation.

\institute[] % (optional, but mostly needed)
{
  Department of Computer Science\\
  Geoorgia State University
}
 
% - Use the \inst command only if there are several affiliations.
% - Keep it simple, no one is interested in your street address.

\date[Short Occasion] % (optional)
{2/20/2014}

\subject{Talks}
% This is only inserted into the PDF information catalog. Can be left
% out. 



% If you have a file called "university-logo-filename.xxx", where xxx
% is a graphic format that can be processed by latex or pdflatex,
% resp., then you can add a logo as follows:

% \pgfdeclareimage[height=0.5cm]{gsu-logo}{gsu-logo}
% \logo{\pgfuseimage{gsu-logo}}



% Delete this, if you do not want the table of contents to pop up at
% the beginning of each subsection:



% If you wish to uncover everything in a step-wise fashion, uncomment
% the following command: 

%\beamerdefaultoverlayspecification{<+->}


\begin{document}

\begin{frame}
  \titlepage
\end{frame}

\begin{frame}{Outline}
  \tableofcontents
  % You might wish to add the option [pausesections]
\end{frame}


% Since this a solution template for a generic talk, very little can
% be said about how it should be structured. However, the talk length
% of between 15min and 45min and the theme suggest that you stick to
% the following rules:  

% - Exactly two or three sections (other than the summary).
% - At *most* three subsections per section.
% - Talk about 30s to 2min per frame. So there should be between about
%   15 and 30 frames, all told.

\section{Introduction to DNS Security}

\subsection[DNS]{What is DNS?}


\subsection{How is DNS Secured}

\begin{frame}{SSL/TLS}
	Alice wishes to Start a communication with Bob.\\
	Alice already knows the Certificate Authority's Public key: $PK_{C}$\\
	\begin{itemize}
		\item{ A$\Rightarrow{}$B: "ClientHello" // Initiate exchange} 
		\item{ B$\Rightarrow{}$A: $PK_{B}$ signed by $PK_{C}$ // Send the certificate}
		\item{ A$\Rightarrow{}$B: $E(PK_{B}$, NONCE) // Sends a key for the stream cipher } 
	\end{itemize}

\end{frame}

\begin{frame}{Certificate Authorities}
\end{frame}

\begin{frame}{DNSSEC}
\end{frame}

\section{DNS Adversary Model}

\subsection{The Byzentine Generals Problem}

\begin{frame}{Key Exchange is Key}
\end{frame}

\begin{frame}{Trusted Third Parties solve everything}
\end{frame}

\begin{frame}{Trusted Third Parties are centralized points of failure}
\end{frame}

\section{Proof of Work Chain based Key Distribution}

\begin{frame}{What is a Proof of Work}
\end{frame}

\begin{frame}{How does Bitcoin Work?}
\end{frame}

\begin{frame}{A New SSL Protocol}
	Alice wishes to Start a communication with Bob.\\
	Alice has setup a CA with Public key: $PK_{C}$\\
	\begin{itemize}
		\item{ A$\Rightarrow{}$B: "ClientHello" // Initiate exchange} 
		\item{ B$\Rightarrow{}$A: $PK_{B}$ // send B's public key}
		\item{ A$\Rightarrow{}$CA: $E(PK_{CA}$, NONCE) // Sends a key for the stream cipher to teh CA} 
	\end{itemize}


\end{frame}

\begin{frame}{Modifcations to Bitcoin for a key exchange}
\begin{itemize}
	\item{Limit the length of the blockchain to 1 year}
	\item{Add a 'physical' layer check to authenticate new transactions}
	\item{Consider alternative incentive methods}
\end{itemize}
\end{frame}

\section{Future Research Directions}

\begin{frame}{Research Directions}
\begin{itemize}
	\item{Imropoved SSL}
	\item{New DNS distribution Options}
	\item{Improved Software Lisensing}
	\item{Greater avalibility of PGP style messaging}
	\item{Just about anything that needs a key exchange}
	
\end{itemize}


\end{frame}





\end{document}


